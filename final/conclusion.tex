\textcolor[rgb]{1.00,0.00,0.00}{We present a data-driven approach for control-oriented modeling of large-scale cyber-physical systems. We show how regression trees are well suited to address challenges associated with Demand-Response for large \textit{C/I/I} consumers while being interpretable. We present a data-driven predictive control with regression trees (DPCRT): an algorithm for implementing receding horizon control with regression trees-based models. DPCRT enables the use of receding horizon control synthesis for complex problems, which otherwise are dependent on first principles based models. We applied DPCRT to the peak power reduction problem in buildings. The performance of DPCRT are evaluated on a DoE commercial reference virtual test-bed. It results in a much lower energy consumption when compared to the Naive control strategy. DPCRT also leads to $8.6\%$ decrease in the peak power reduction of the building when compared to mbCRT and $3.1\%$ when compared to the Naive peak reduction strategy. These advantages combined with the fact that the tree-based methods achieve high prediction accuracy, make DPCRT an alluring tool for evaluating and planning DR curtailment responses for large scale cyber-physical energy systems.}

\textcolor[rgb]{0.00,0.00,1.00}{\textbf{\emph{On-going work.}} The approach we propose in this paper modifies the CART algorithm considering the error minimization of the system evolution over a predictive horizon. This results in a multi-output regression tree-based system modeling that is extremely simple from the computational complexity point of view and that give good accuracy for the prediction. New methodologies are currently under investigation that consider multiple trees with single output. Preliminary results can be found in \cite{JainACC2017,JainCDC2017}.}
