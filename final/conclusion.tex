We present a data-driven approach for control-oriented modeling of large scale cyber-physical energy systems.
We show how regression tree based methods are well suited to address challenges associated with demand response for large \textit{C/I/I} consumers while being interpretable. 
%We have incorporated all our methods into the DR-Advisor tool - \url{http://mlab.seas.upenn.edu/dr-advisor/}.
DR-Advisor achieves a prediction accuracy of upto $\textbf{98.9\%}$ for eight buildings on the University of Pennsylvania's campus.
We compare the performance of DR-Advisor on a benchmarking data-set from AHRAE's energy predictor challenge and rank $2^nd$ among the winners of that competition.
%We show how DR-Advisor can select the best rule-based DR strategy, which leads to the most amount of curtailment, from a set of several rule-based strategies. 
We presented a model based control with regression trees (mbCRT) algorithm which enables control synthesis using regression tree based structures for the first time. Using the mbCRT algorithm, DR-Advisor can achieve a sustained curtailment of $\textbf{380kW}$ during a DR event and a revenue of $\sim\$\textbf{45,600}$ over one summer.
The mbCRT algorithm outperforms even the best rule-based strategy by $\textbf{17\%}$.

We also present a data predictive control with regression trees (DPCRT) is an algorithm for implementing receding horizon control with regression trees based models. 
DPCRT enables the use of receding horizon control synthesis for problems of peak power reduction in buildings which otherwise are dependent on first principles based model of the building.
The performance of DPCRT is evaluated on a DoE commercial reference virtual test-bed. DPCRT results in a much lower energy consumption when compared to a Naive control strategy.
DPCRT also leads to \textbf{8.6\%} decrease in the peak power reduction of the building as compared to mbCRT and \textbf{3.1\%} as compared to the Naive peak reduction strategy.
%DR-Advisor bypasses cost and time prohibitive process of building high fidelity models of buildings that use grey and white box modeling approaches while still being suitable for control design.
These advantages combined with the fact that the tree based methods achieve high prediction accuracy, make DR-Advisor an alluring tool for evaluating and planning DR curtailment responses for large scale cyber-physical energy systems.


