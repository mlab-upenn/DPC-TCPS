We present a data-driven approach for control-oriented modeling of large-scale cyber-physical systems. 
%We show how regression trees are well suited to address challenges associated with Demand-Response for large \textit{C/I/I} consumers while being interpretable. 
We present a data-driven predictive control with regression trees (DPCRT) -- an algorithm for implementing receding horizon control. 
DPCRT enables optimal control designs for complex problems which otherwise are dependent on expensive first principles or physics-based models. 
We apply DPCRT to the problem of peak power reduction in buildings in context of Demand Response. 
The performance of DPCRT are evaluated on a DoE commercial reference virtual test-bed. 
%It results in a much lower energy consumption when compared to the Naive control strategy. 
DPCRT leads to $8.6\%$ decrease in the peak power consumption of the building when compared to the mbCRT algorithm (one-step look ahead) and $3.1\%$ decrease when compared to a Naive rule-based reduction strategy. 
These advantages make DPCRT an alluring tool for evaluating and planning DR curtailment responses for large scale cyber-physical energy systems.

\textbf{\emph{On-going work.}} The approach we propose in this paper modifies the CART algorithm considering the error minimization of the system evolution over a predictive horizon. 
This results in a multi-output regression tree modeling that is extremely simple from the computational complexity point of view. 
To improve the accuracy of our algorithms, new methods based on Random Forests are currently under investigation. Preliminary results can be found in \cite{JainACC2017,JainCDC2017}.
