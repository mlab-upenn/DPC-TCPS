There is a vast amount of literature like (\cite{oldewurtel2013towards,xu2004peak}) which addresses the problem of determining demand response strategies. 
The majority of approaches are using either rule-based approaches for curtailment or white/grey box model-based approaches.
These usually assume that the model of the system is either perfectly known or found in literature, whereas the task is much more complicated and time consuming in case of a real building and sometimes, it can be even more complex and involved than the controller design itself.
After several years of work on using first principles based models for demand response, multiple authors~\cite{costmpc,reallife} have concluded that the biggest hurdle to mass adoption of intelligent building control is the cost and effort required to capture accurate dynamical models of the buildings.
Since DR-Advisor learns an aggregate building level models and combined with the fact that weather forecasts are expected to become cheaper; there is little to no additional sensor cost of implementing the DR-Advisor recommendation system in large buildings. OpenADR standard and protocol~\cite{openadr} describes the formats for information exchange to facilitate DR but modeling, prediction and control strategies are out of scope.
There are ongoing efforts to make tuning and identifying white box models of buildings more autonomous~\cite{new2012autotune}.
Figuring out the correct response on a fast time scales (1-5 mins) using just data-driven methods has'nt been adequately addressed before and makes the DR-Advisor approach and tool novel.

Several machine learning approaches~\cite{edwards2012predicting,vaghefi2014modeling,yin2012scalable} have been utilized before for forecasting electricity load including some which use regression trees. 
However, there are two significant shortcomings of the work in this area: 
\begin{inparaenum}[(a)]
\item First, these approaches are coarse grained and  are not aimed at solving demand response problems but are restricted to long term load forecasting with applications in evaluating building retrofits savings and building energy ratings. 
\item Secondly, there is no focus on control synthesis or addressing the suitability of the model to be used in control design; whereas the mbCRT algorithm enables the use of regression trees for control synthesis with applications in demand response. 
\end{inparaenum}